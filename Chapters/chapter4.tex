\documentclass[../main.tex]{subfiles}
\begin{document}

\chapter{Robot path visualisation}
\label{chp:robotpathvisual}

\section{Qt Framework}

Qt is a C++ framework for the development of native desktop applications.

For the purpose of testing the NURBS library and its functionalities, a program
was created using Qt and C++.

The actual visualization of NURBS paths was done using the QGraphicsView module.
This module implements various drawing functions including those for drawing
line segments and points.

The class \texttt{qCurve}, which inherits from \texttt{QGraphicsObject}, was
created as a wrapper around the \texttt{Curve} class to provide Qt drawing
functionality to a NURBS curve.

Overloading the \texttt{paint} function of \texttt{QGraphicsObject} allows us to
define how we want our object to be drawn in a \texttt{QGraphicsView} widget. In
the case of NURBS curves, this means that it is possible to draw not just the
curve itself, but other data associated with it, such as its control points,
knots, or its bounding box. The \texttt{polyline} method defined in the NURBS
library creates a sequence of points for the express purpose of quick and simple
visualization. On every scene update each curve's polyline is recomputed and
drawn using the \texttt{drawLine} method of the \texttt{QPainter} class.

The open-source Qt widget QCustomPlot is used to visualize the basis functions
of the selected curve\footnote{Available at \url{https://www.qcustomplot.com/}}.
The plotting is done by taking each span's basis matrix and sampling it over an
interval of $[0,1]$, adjusted to the knot span's actual beginning and end in the
knot vector.

\section{RViz}

For testing in a simulation environment a plugin was created for the RViz tool
often used for visualization of robots.

RViz listens for ROS messages of a given type and, upon receiving data, draws
the data appropriately. In this case, the message is a Path with timestamps.

\end{document}