\documentclass[../main.tex]{subfiles}
\begin{document}

\chapter{Applications and improvements}
\label{chp:improvements}

The library was created for the purpose of top-down path mapping, and as such
works only in 2D space. As it stands it can be used for any 2D application,
including controlling robot manipulators, e.g. in plotting or CNC machines.

One possible improvement to the library would be the ability to define NURBS
curves in any dimension, including 1D and from 3D onward. This could be achieved
using Eigen's class templates which can define the dimensions of matrices at
compile time, in this case the width of the control point matrix and the
\texttt{\_cached\_vbf} matrix. However, expanding the library to NURBS surfaces
in 2-dimensional \emph{parametric} space (parameters $t$ and $v$) would require
significant refactoring as all the methods rely on a scalar $t$ value. 

Three-dimensional splines can be used to define precise paths for industrial
robot manipulators.

One-dimensional splines have limited uses, however one possible application
would be to directly control the joints of a robot with precise control over its
timing and speed.

The library was written using the \emph{C++17} standard, meaning that it misses
out on some features of modern C++. Additionally, during development Eigen was
updated to version \emph{5.0.0} and ROS2 has become the standard in favor of
ROS1, which is slowly being deprecated. Using modern standards could result in
better optimization and performance.

\end{document}