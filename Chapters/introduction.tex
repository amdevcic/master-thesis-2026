\documentclass[../main.tex]{subfiles}
\begin{document}

\chapter*{Introduction}
\label{chp:introduction}
\addcontentsline{toc}{chapter}{Introduction}

A rapid increase in online commerce has created a need for more automation in
logistics. Many manufacturers and warehouses have started relying on fully or
almost fully automated mobile robots instead of human labor, largely due to
safety concerns and labor shortage, as well as a need for greater efficiency and
precision than human labor can provide.

Traditionally, mobile robots, especially ones used for material handling in
logistics and manufacturing, have been classified into two categories, automated
guided vehicles (AGVs) and autonomous mobile robots (AMRs). AGVs operate on
predefined paths while AMRs are free to plan their own path for every new
mission. AGV-s often have a controller system which commands and controls all
robots in the network. In some cases AMR-s might be the solution that requires
less human intervention and input and may be more efficient, however AGV-s allow
for more control and predictability. The line separating these two approaches is
becoming increasingly blurred as some AGVs can now re-plan their paths around
obstacles. This thesis is focused on pre-planned paths and will refer to the
vehicles simply as mobile robots.

\end{document}