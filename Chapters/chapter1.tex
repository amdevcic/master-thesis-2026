\documentclass[../main.tex]{subfiles}
\begin{document}

\chapter{Robot vehicle path design}
\label{chp:robotpathdesign}

Automated guided vehicles, or AGV-s for short, are mobile robots often used in
logistics and other industries. They differ from autonomous mobile robots, or
AMR-s, in the fact that AGV-s operate on predefined paths while AMR-s operate
with no guidance. AMR-s do not have predefined paths, they use artificial
intelligence and sensors and communicate with other robots in the network to
plan out their own optimal paths. AGV-s often have a controller system which
commands and controls all robots in the network. In some cases AMR-s might be
the solution that requires less human intervention and input and may be more
efficient, however AGV-s allow for more control and predictability.

In the past, AGV-s used physical rails or magnetized or induction tracks placed
in the environment to guide their movement. Modern AGV-s have their paths
defined in a controller application which sends commands to dictate their
movement. While driving, the robots use various sensors to verify their speed
and position and correct their path if necessary. Some AGV-s require reflective
stickers for navigation, though in other cases no changes need to be made to the
environment. In that case the AGV uses a combination of LIDAR sensors and
computer vision to verify their position in space and their speed.

\section{Path continuity}

Two mathematical concepts relevant to AGV navigation are geometric continuity
and parametric continuity. Parametric continuity of the n-th degree, or $C^n$,
is achieved when the parameters of the n-th derivative of a function are
continuous. Geometric continuity of the n-th degree, or $G^n$, is a more relaxed
form of parametric continuity, and is achieved if a function can be
reparametrized as a $C^n$ continuous function.

\end{document}