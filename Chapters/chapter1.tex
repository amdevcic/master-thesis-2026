\documentclass[../main.tex]{subfiles}
\begin{document}

\chapter{Mobile robot path design}
\label{chp:robotpathdesign}

In the past, AGV-s used physical rails, or magnetized or induction tracks,
placed in the environment to guide their movement. Modern AGV-s have their paths
defined in a controller application which sends commands to dictate their
movement. While driving, the robots use various sensors to verify their
velocity, position and orientation  and correct their path if necessary. Some
AGV-s require reflective stickers for navigation, though in other cases no
changes need to be made to the environment. In that case the AGV uses a
combination of LIDAR sensors and computer vision to verify their position in
space and their speed.

\begin{figure}[h]
    \centering
    \includegraphics[width=0.7\linewidth]{roadmap_cropped.jpg}
    \label{fig:roadmap}
    \caption{Example of a roadmap.}
\end{figure}

\section{Path continuity}

Two mathematical concepts relevant to mobile robot navigation are geometric
continuity and parametric continuity of the robot's path. Parametric continuity
of the n-th degree, or $C^n$, is achieved when the parameters of the n-th
derivative of a function are continuous. Geometric continuity of the n-th
degree, or $G^n$, is a more relaxed form of parametric continuity, and is
achieved if a function can be reparametrized as a $C^n$ continuous function.
\cite{geometric_cont}

A $G^0$ continuous spline is one where all of the points on the curve are
continuous, whereas a $G^1$ continuous spline has a continuous tangent direction
as well as position. A spline with $G^2$ continuity may be appropriate for most
uses. It has continuity in position, speed and acceleration.

For the navigation of multi-wheeled mobile robots it is important to take into
account the \emph{jerk}, or the rate of change of acceleration. A curve where
the acceleration changes continuously would be considered to have $G^3$
continuity.
\cite{kokot2021arxiv}

A curve appropriate for mobile robot path design needs to have three attributes:
it must have geometric continuity of at least the third order, it must be easy
to manipulate to maximize the usage of available space, and it must be able to
be shaped in such a way to minimize the derivative of the curvature.

The curvature derivative is important because it indicates how much and how
sharply the mobile robot must turn its steering wheels. A high curvature
derivative means that the robot needs to rotate its steering wheel a greater
angle in a shorter amount of time, which may put strain on the robot and be
dangerous for the people around it, especially if carrying heavy or unstable
cargo. In addition, robots have a physical limit as to how quickly they can turn
based on their size and architecture.



\end{document}