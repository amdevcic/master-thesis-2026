\documentclass[../main.tex]{subfiles}
\begin{document}

\chapter{Mobile robot path design}
\label{chp:robotpathdesign}

Automated guided vehicles, or AGV-s for short, are mobile robots often used in
logistics and other industries. They differ from autonomous mobile robots, or
AMR-s, in the fact that AGV-s operate on predefined paths while AMR-s operate
with no guidance. AMR-s do not have predefined paths, they use artificial
intelligence and sensors and communicate with other robots in the network to
plan out their own optimal paths. AGV-s often have a controller system which
commands and controls all robots in the network. In some cases AMR-s might be
the solution that requires less human intervention and input and may be more
efficient, however AGV-s allow for more control and predictability.

In the past, AGV-s used physical rails or magnetized or induction tracks placed
in the environment to guide their movement. Modern AGV-s have their paths
defined in a controller application which sends commands to dictate their
movement. While driving, the robots use various sensors to verify their speed
and position and correct their path if necessary. Some AGV-s require reflective
stickers for navigation, though in other cases no changes need to be made to the
environment. In that case the AGV uses a combination of LIDAR sensors and
computer vision to verify their position in space and their speed.

\section{Path continuity}

Two mathematical concepts relevant to mobile robot navigation are geometric
continuity and parametric continuity. Parametric continuity of the n-th degree,
or $C^n$, is achieved when the parameters of the n-th derivative of a function
are continuous. Geometric continuity of the n-th degree, or $G^n$, is a more
relaxed form of parametric continuity, and is achieved if a function can be
reparametrized as a $C^n$ continuous function. \cite{geometric_cont}

A $G^0$ continuous spline is one where all of the points on the curve are
continuous, whereas a $G^1$ continuous spline has a continuous tangent direction
as well as position. A spline with $G^2$ continuity may be appropriate for most
uses. It has continuity in position, speed and acceleration.

For the navigation of multi-wheeled mobile robots it is important to take into
account the \emph{jerk}, or the rate of change of acceleration. A curve where
the acceleration changes continuously would be considered to have $G^3$
continuity.
\cite{kokot2021arxiv}

A curve appropriate for mobile robot path design needs to have three attributes:
it must have geometric continuity of at least the third order, it must be easy
to manipulate to maximize the usage of available space, and it must be able to
be shaped in such a way to minimize the derivative of the curvature.

\section{Robot Operating System}
The \emph{Robot Operating System}, or ROS for short, provides a standardized
interface between user programs and robot architecture, and is commonly used in
the field of robotics. 

ROS is mainly built upon the concept of \emph{publishers} and \emph{subscribers}
sending and receiving data among each other. A publisher sends \emph{messages},
which can contain various different types of data, to a \emph{topic}. These will
often contain user commands, sensor data or other real-time information that a
robot might use. A subscriber subscribes to a topic and receives all messages
sent to that topic. Subscribers may receive commands for a robot, but also
collect data from a robot's sensors to show to the user.

\section{Visualization of AGV paths}


\end{document}