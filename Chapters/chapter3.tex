\documentclass[../main.tex]{subfiles}
\begin{document}

\chapter{NURBS curve implementation}
\label{chp:nurbsimplementation}

For the purpose of path planning for autonomous guided vehicles, a C++ library
was created for the implementation and manipulation of NURBS curves. The library
utilizes the Eigen library for its linear algebra functions.

The library implements all necessary functions for creating and using NURBS
curves. Elementary transformations, such as translation and rotation, are not
implemented. It is important to note one of the properties of NURBS curves which
is that all transformations applied to the set of control points are equivalent
to transformations on the curve itself, meaning that the end-user may implement
these functions themselves if necessary. The library only provides the minimum
required functions to cover all operations exclusive to NURBS curves.

\section{Class structure}

The main data structure that is available to the user is the \texttt{Curve} class. A
curve is divided into knot spans, or segments of the curve between two adjacent
knots, which are represented by the \texttt{Span} utility class. A Curve object manages
its own Span objects which are not visible outside of the class, nor is the Span
class itself visible to the end user.

\subsection{Curve}

The \texttt{Curve} class represents an array of continuous spans.

\subsection{Span}

The \texttt{Span} class defines a segment of a curve between two knots. It
shares its order with its parent curve and the number of control points is
limited by its order. A single span does not have a knot vector. The shape of
a span is defined by its basis function, which is in turn defined by the parent
curve's knot vector. This basis function is defined in the code as a 4 by 4
matrix.

For the sake of performance and memory efficiency, the \texttt{Span} class does
not store its control points, but instead caches the dot product of the weights
with the basis function, as well as the weighted control points with the basis
function.


\section{Key method implementations}

\subsection{Sampling methods}


\subsection{Operations between two curves}

\end{document}