\documentclass[../main.tex]{subfiles}
\begin{document}

\chapter{NURBS curve implementation}
\label{chp:nurbsimplementation}

For the purpose of path planning for autonomous guided vehicles, a C++ library
was created for the implementation and manipulation of NURBS curves. The library
utilizes the Eigen library for its linear algebra functions.

The main data structure that is available to the user is the Curve class. A
curve is divided into knot spans, or segments of the curve between two adjacent
knots, which are represented by the *Span* utility class. A Curve object manages
its own Span objects which are not visible outside of the class, nor is the Span
class itself visible to the end user.

The library implements all necessary functions for creating and using NURBS
curves. Elementary transformations, such as translation and rotation, are not
implemented It is important to note one of the properties of NURBS curves which
is that all transformations applied to the set of control points are equivalent
to transformations on the curve itself, meaning that the end-user may implement
these functions themselves if necessary. The library only provides the minimum
required functions to cover all operations exclusive to NURBS curves.

\end{document}