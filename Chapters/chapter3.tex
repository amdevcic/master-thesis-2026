\documentclass[../main.tex]{subfiles}
\begin{document}

\chapter{NURBS curve implementation}
\label{chp:nurbsimplementation}

For the purpose of path planning for autonomous guided vehicles, a C++ library
was created for the implementation and manipulation of NURBS curves. The library
utilizes the Eigen library for its linear algebra functions.

The library implements all necessary functions for creating and using NURBS
curves. Elementary transformations, such as translation and rotation, are not
implemented. It is important to note one of the properties of NURBS curves which
is that all transformations applied to the set of control points are equivalent
to transformations on the curve itself, meaning that the end-user may implement
these functions themselves if necessary. The library only provides the minimum
required functions to cover all operations exclusive to NURBS curves.

\section{The Eigen library}

\emph{Eigen} is an open-source C++ template library for linear algebra
\cite{eigenweb}. It is commonly used in the fields of computer graphics and
robotics. The library provides class and function templates that allow for
compile-time optimization, as well as \emph{SIMD} ("single instruction, multiple
data") vectorization which drastically speeds up operations on vectors and
matrices. Using \emph{Eigen's} built-in vector and matrix classes allows for
interoperability with other libraries and end-user programs that use
\emph{Eigen}, which is often the case in robotics.

\section{Class structure}

The main data structure that is available to the user is the \texttt{Curve} class. A
curve is divided into knot spans, or segments of the curve between two adjacent
knots, which are represented by the \texttt{Span} utility class. A Curve object manages
its own Span objects which are not visible outside of the class, nor is the Span
class itself visible to the end user.

\subsection{Curve}

The \texttt{Curve} class represents a complete NURBS curve, i.e. sequence of
continuous knot spans. It stores some of the basic data of a particular curve,
including its control points, knot vector, and order. Internally it is
represented by an array of \emph{Span} objects containing the data required to
evaluate many NURBS operations, though they are not accessible to the end-user.

Methods that operate on a single span rather than the entire curve involve
evaluating which spans are being affected, then propagating the function call to
the \emph{Span} objects themselves. The \texttt{getKnotSpanIndex(t)} method
returns the index of the knot span containing the value $t$. The result is
always a single span and never a set of spans. If the value $t$ is itself a knot
with a multiplicity greater than one, the function returns the first span after
the multiple knot, and never the zero-length span containing the knot $t$.

\subsection{Span}

The \texttt{Span} class defines a segment of a curve between two knots. It
shares its order with its parent curve and the number of control points is
limited by its order. A single span does not have a knot vector. The shape of
a span is defined by its basis function, which is in turn defined by the parent
curve's knot vector. This basis function is defined in the code as a 4 by 4
matrix and is stored in each span.

For the sake of performance and memory efficiency, the \texttt{Span} class does
not store its control points, but instead caches the dot product of the weights
with the basis function, as well as the weighted control points with the basis
function, as required by Equation \eqref{eq:nurbs_point} for the evaluation of a
point on the curve.

Whenever the curve's knot vector is modified, the affected spans recompute their
basis functions as defined in Equation \eqref{eq:basisfunc}. Afterwards they
recompute their cached control point and weight matrices.

\section{Key method implementations}

\subsection{Sampling methods}

The \texttt{valueAt} method directly implements equation \eqref{eq:nurbs_point}
using matrix notation. The helper function \texttt{\_powSeries(t, p)} computes the
powers of $t$ up to $p$.

The \texttt{polyline} method creates a sequence of line segments suitable for
drawing or other purposes when only an approximation of the curve is needed.


\subsection{Operations between two curves}

\end{document}