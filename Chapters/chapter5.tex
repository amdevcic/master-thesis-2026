\documentclass[../main.tex]{subfiles}
\begin{document}

\chapter{Testing and results}
\label{chp:results}

\section{NURBS library}

To test if the library is truly suited for navigation of mobile robots, a series
of tests was made on a fourth degree curve with a uniform knot vector, as
defined in Table \ref{tab:points}. A short C++ program was created to sample the
curve at a series of uniformly spaced parameter values, and the resulting data
was plotted using Python and the \texttt{matplotlib} library. If the methods for
evaluating and differentiating NURBS curves are implemented correctly, the
following should be true: the first three derivatives are continuous, indicating
G3 continuity, and the curvature and curvature derivative should be continuous.

\begin{table}[h]
\centering
\begin{tabular}{ccc}
\hline
$x$ & $y$ & $w$ \\
\hline
0  & 0  & 1.0 \\
10 & 10 & 1.0 \\
0  & 20 & 1.0 \\
10 & 30 & 1.0 \\
0  & 40 & 1.0 \\
10 & 50 & 1.0 \\
\hline
\end{tabular}
\caption{Control points for test curve}
\label{tab:points}
\end{table}

The curve's first three derivatives were uniformly sampled across a hundred
points from $0.0$ to $1.0$ using the \texttt{derivativeAt} method of the
\texttt{Curve} class. The partial derivatives on the $x$ and $y$ axes can be
seen on Figures \ref{fig:test_d1}, \ref{fig:test_d2} and \ref{fig:test_d3}. The
test curve is split into two knot spans at $t=0.5$, which is shown on the plots
as a dashed vertical line.

\begin{figure}[h]
    \centering
    \includegraphics[width=0.7\linewidth]{d1.png}
    \caption{The first derivative, with $x$ in red and $y$ in blue.}
    \label{fig:test_d1}
\end{figure}

\begin{figure}[h]
    \centering
    \includegraphics[width=0.7\linewidth]{d2.png}
    \caption{The second derivative, with $x$ in red and $y$ in blue.}
    \label{fig:test_d2}
\end{figure}

\begin{figure}[h]
    \centering
    \includegraphics[width=0.7\linewidth]{d3.png}
    \caption{The third derivative, with $x$ in red and $y$ in blue.}
    \label{fig:test_d3}
\end{figure}    

It is obvious from Figure \ref{fig:test_d3} that, while the third derivative is
indeed continuous, it has a sharp peak in the middle, meaning that the fourth
derivative will not be continuous. This aligns with the fact that a $p$-th
degree curve will have continuity of order $p-1$. Each knot span's derivative is
a polynomial which is one degree lower than the previous.

The curvature and its derivative were uniformly sampled along the curve along
with the derivatives, using the \texttt{curvatureAt} and
\texttt{curvatureDerivativeAt} methods. The result can be seen in Figure
\ref{fig::curvature_plot}.

\begin{figure}[h]
    \centering
    \includegraphics[width=0.7\linewidth]{curvature.png}
    \caption{The curvature $\kappa$ in teal and its derivative in orange.}
    \label{fig::curvature_plot}
\end{figure} 

The \texttt{polyline} method was tested along with the derivatives and
curvature. For reference, the \texttt{valueAt} method was uniformly sampled
along the same points as the previous tests. Figure \ref{fig:polyline_curvature}
shows a side-by-side comparison of the two methods and their vertices, with the
$x$ and $y$ axes swapped for clarity. Both curves appear visually identical but
have different point densities. To illustrate the efficiency of both methods,
the absolute value of the curvature at the given points is displayed using a
color ramp. The naive method with uniformly sampled points, shown on the bottom,
has more points overall. The middle portion has equally dense points on areas
with high curvature and low curvature, while the ends of the curve are more
sparse due to the fact that we are sampling by parameter values, not by length.
The \texttt{polyline} method, on the other hand, has fewer points overall.
Despite the method not taking the curvature formula into account, there is a
clear correlation between point density and curvature. The ends of the curve are
in fact more dense than the uniformly sampled method, while the middle has
relatively few points. Uniform sampling in larger intervals may lead to fewer
points along straight sections, but it would also result in sparse points along
high-curvature segments.

\begin{figure}[h]
    \centering
    \includegraphics[width=\linewidth]{polyline_curvature.png}
    \caption{Comparison of two polyline algorithms. (Top) \texttt{polyline}
    method. (Bottom) Uniform sampling of $t$.}
    \label{fig:polyline_curvature}
\end{figure}

\section{RViz plugin}

Multiple configurations of NURBS curves were tested in the RViz plugin to test
its accuracy. The configurations can be seen in Listing \ref{lst:yaml-configs}.
For the sake of clarity only the data relevant to NURBS curves is listed. The
message uses the $z$ coordinate to define weights. The reasoning behind this
design choice was that the control points are defined in homogenous space and
projected onto the plane $w=1$.

In order to run ROS and RViz locally, they need to be installed on an
\emph{Ubuntu} system. The setup for testing with RViz requires three programs.
In one terminal the \texttt{roscore} command must be executed and left running,
which runs the ROS server that connects all nodes. The second terminal needs to
run the \texttt{rviz} command to run RViz. The third terminal is used to send
messages via the \texttt{rostopic} command.

Having built the custom message types and the plugin, RViz must be set up to
track messages of type \texttt{PathStamped} on a topic with an arbitrary name.
Then, by running \texttt{rostopic pub [/topic\_name] PathStamped [.yaml data]}
the message gets sent to the topic and read by its subscriber, RViz. Once it
receives the message it will be displayed as a pink line. The viewport can be
moved around and the line properties can be edited. 

\begin{listing}[H]
    \centering
    \begin{minipage}[t]{0.3\textwidth}
        \begin{minted}[fontsize=\scriptsize, linenos, baselinestretch=0.8]{yaml}
degree: 2
weighted_points:  
- 
    x: 0.0
    y: 0.0
    z: 1.0
- 
    x: 3.0
    y: 10.0
    z: 1.0
- 
    x: 10.0
    y: 3.0
    z: 1.0
-   
    x: 10.0
    y: 10.0
    z: 1.0
knot_vector:
    - 0.0
    - 0.0
    - 0.0
    - 0.5
    - 1.0
    - 1.0
    - 1.0
        \end{minted}
    \end{minipage}%
    \hfill
    \begin{minipage}[t]{0.3\textwidth}
        \begin{minted}[fontsize=\scriptsize, linenos, baselinestretch=0.8]{yaml}
degree: 2
weighted_points: 
- 
    x: 0.0
    y: 0.0
    z: 1.0
- 
    x: 6.0
    y: 20.0
    z: 2.0
- 
    x: 20.0
    y: 6.0
    z: 2.0
-   
    x: 10.0
    y: 10.0
    z: 1.0
knot_vector:
    - 0.0
    - 0.0
    - 0.0
    - 0.5
    - 1.0
    - 1.0
    - 1.0
        \end{minted}
    \end{minipage}%
    \hfill
    \begin{minipage}[t]{0.3\textwidth}
        \begin{minted}[fontsize=\scriptsize, linenos, baselinestretch=0.8]{yaml}
degree: 2
weighted_points: 
- 
    x: 0.0
    y: 0.0
    z: 1.0
- 
    x: 3.0
    y: 10.0
    z: 1.0
- 
    x: 30.0
    y: 9.0
    z: 3.0
-   
    x: 10.0
    y: 10.0
    z: 1.0
knot_vector:
    - 0.0
    - 0.0
    - 0.0
    - 0.1
    - 1.0
    - 1.0
    - 1.0
        \end{minted}
    \end{minipage}
    \caption{NURBS test configurations in YAML format: (left) no changes, (middle) with weights, (right) with weights and knot vector}
    \label{lst:yaml-configs}
\end{listing}

\begin{figure}[h]
    \centering
    \includegraphics[width=\linewidth]{normal.png}
    \caption{RViz plugin displaying a uniform, unweighted NURBS curve}
\end{figure} 

\begin{figure}[h]
    \centering
    \includegraphics[width=\linewidth]{weights.png}
    \caption{RViz plugin displaying a uniform, weighted NURBS curve}
\end{figure} 

\begin{figure}[h]
    \centering
    \includegraphics[width=\linewidth]{weight_knot.png}
    \caption{RViz plugin displaying a non-uniform, weighted NURBS curve}
\end{figure} 

\end{document}