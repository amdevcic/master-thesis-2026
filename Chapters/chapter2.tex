\documentclass[../main.tex]{subfiles}
\begin{document}

\chapter{Parametric curves and splines}
\label{chp:parametriccurves}

Parametric curves are curves defined in a vector space by a parametric equation,
which can be as simple as a parabola or a circle. In an n-dimensional space,
these functions are often defined by one input — most commonly $t$ for time,
limited to the interval $[0, 1]$ — and $n$ outputs. Each of the $n$ coordinates
is its own independent function, and the combination of multiple coordinates
creates a curve, plane, or hyperplane.

Splines are a special kind of parametric curve that are defined as piecewise
polynomials and are most often used to interpolate between a set of points.

For the purpose of AGV path design, the most appropriate types of splines are
Bezier curves, B-splines, and NURBS curves. What these three models have in
common is that each curve is defined by its order, a set of control points in
n-dimensional space and its basis functions. The position for a given t is a
weighted sum of the curve's points, the weights being given by its basis
functions. In other words, the curve interpolates between all of its given
points.

A curve appropriate for AGV path design needs to have three attributes: it must
have geometric continuity of at least the third order, it must be easy to
manipulate to maximize the usage of available space, and it must be able to be
shaped in such a way to minimize the derivative of the curvature.

\section{Bezier}

Bezier curves are the simplest of the three. The most commonly used types of
Bezier curves are quadratic and cubic curves, defined by three and four points
respectively. The order of a Bezier curve is always equal to the number of
control points minus one.

A Bezier curve's basis functions are known as Bernstein polynomials. There are
$n$ basis functions for a curve of the n-th degree. 

\section{B-spline}

B-splines, short for “basis splines,” are a generalization of Bezier curves.
Unlike Bezier curves, which have static basis functions dependent on order, the
basis functions of a B-spline can be modified using what is called a knot
vector. The number of control points on a B-spline is arbitrary and not
dependent on its order, unlike Beziers.

The basis functions of a B-spline may be equal to zero in the range $(0, 1)$,
meaning that a given basis function may not have influence on the entire curve,
i.e. a given point on the curve may not be defined by all of its control points.
This makes it possible to create complex shapes using a single B-spline.

A B-spline of degree $p$ and with $N$ control points has a knot vector of size
$N+p+1$. Each control point affects $p$ knots. By moving a knot's value up or
down, we can make a section of the curve denser in control points while making
the rest more sparse. Each span between two knots, i.e. between two values of
$t$, can be seen as a separate curve controlled by $p+1$ points, much like a
Bezier. However, this is a much better solution than chaining Bezier curves
because the model guarantees geometric continuity between knot spans, while a
chain of Beziers needs to be adjusted manually to be continuous.

This is very close to what we want but it's still difficult to make sharp turns
and precise movements without increasing the order of the curve, making it more
expensive to compute in real-time. It is possible to achieve even more control
over the shape of our curve using NURBS.

\section{NURBS}

NURBS curves are a further generalization of B-splines. The acronym stands for
“non-uniform rational basis spline.”

We describe NURBS as “non-uniform” because its basis functions do not need to be
uniformly distributed. This is done using a knot vector which defines the
distribution of a curve's basis functions. B-splines are also non-uniform,
though this is not reflected in the name.

We also describe NURBS as “rational” because they let us define a ratio, i.e. a
weight, for each control point. This means that we can define the influence that
a control point has on the overall curve. In terms of basis functions, this
effectively scales a particular basis function in relation to its neighboring
functions.

Both of the aforementioned parametric curve models are subsets of NURBS curves.
By limiting the parameters of a NURBS curve we can effectively create a Bezier,
a B-spline, a rational Bezier and so on.

Much like B-splines, a NURBS curve of degree p and N control points has a knot
vector of size N+p+1 and each point affects p knots.

A NURBS curve of degree p always has geometric continuity of the p-th degree
along its length. Changing the knot vector does not affect this property unless
there are p identical knots inside the curve (not on the endpoints) which
creates a peak. This should be avoided, or prohibited to the end-user, if the
goal is to maintain geometric continuity. Adjusting the weights does not affect
geometric continuity in any possible case.

Out of the three curve models presented, NURBS are the most flexible and allow
for the most control, though they might be computationally expensive. However,
having modifiable weights gives us a level of flexibility above B-splines
without increasing the order, which makes them potentially more efficient than
B-splines if used wisely.

\end{document}