\documentclass[masterthesis]{fer}
% Add the option upload to generate the final version which is uploaded to FERWeb
% Dodaj opciju upload za generiranje konačne verzije koja se učitava na FERWeb

\usepackage{blindtext}
\usepackage{subfiles}
\usepackage{multicol}

\graphicspath{ {./Images/} }

\usepackage[outputdir=Build]{minted}
\setminted{
    breaklines=true,
    breakanywhere=false,  % Only break at spaces (default)
}

\usepackage{pgfplots}
\usepgfplotslibrary{groupplots}
\usepgfplotslibrary{external}
\tikzexternalize

\pgfplotsset{
    % Global settings for ALL plots
    compat=1.18,
    every axis/.style={
        axis lines = left,
        tick label style={font=\scriptsize},
        label style={font=\footnotesize},
        legend style={font=\footnotesize},
    },
    % Global settings for groupplots specifically
    every axis plot/.style={
        samples=100
    }
}

\newcommand{\citenone}{[?] \PackageWarning{TODO}{Placeholder reference}}

%--- THESIS INFORMATION / PODACI O RADU ----------------------------------------

% Title in English / Naslov na engleskom jeziku
\title{Design and visualisation of mobile robot paths}

% Title in Croatian / Naslov na hrvatskom jeziku
\naslov{Dizajn i vizualizacija putanja robotskih vozila}

% Thesis number / Broj rada
\brojrada{1187}

% Author / Autor
\author{Ana Marija Devčić}

% Mentor 
\mentor{prof.dr.sc. Željka Mihajlović}

% Date in English / Datum rada na engleskom jeziku
\date{February, 2026}

% Date in Croatian / Datum rada na hrvatskom jeziku
\datum{veljača, 2026.}

%-------------------------------------------------------------------------------


\begin{document}


% Titlepage is automatically generated / Naslovnica se automatski generira
% \maketitle


%--- THESIS ASSIGNMENT / ZADATAK -----------------------------------------------

% Thesis assignment is included from external file / Zadatak se ubacuje iz vanjske datoteke
% Enter the filename of the PDF downloaded from FERWeb / Upiši ime PDF datoteke preuzete s FERWeb-a
% \zadatak{filename.pdf}


%--- ACKNOWLEDGMENT / ZAHVALE --------------------------------------------------

\begin{zahvale}
  % Write in the acknowledgment / Ovdje upišite zahvale
  I would like to thank my mentor, dr.sc. Željka Mihajlović, for guiding me
  through my university career, my colleagues at Romb Technologies for
  supporting me in being a better developer, and my friends and loved ones for
  always believing in me. 
\end{zahvale}

% Page numbering starts from here / Odovud započinje numeriranje stranica
\mainmatter

% Table of contents is automatically generated / Sadržaj se automatski generira
\tableofcontents

% Introduction
\subfile{Chapters/introduction.tex}

% Robot vehicle path design
\subfile{Chapters/chapter1.tex}

% Parametric curves and splines
\subfile{Chapters/chapter2.tex}

% NURBS curve implementation
\subfile{Chapters/chapter3.tex}

% Robot path visualisation
\subfile{Chapters/chapter4.tex}

% Results
\subfile{Chapters/chapter5.tex}

% Applications and improvements
\subfile{Chapters/chapter6.tex}

\subfile{Chapters/conclusion.tex}

\subfile{Chapters/acknowledgement.tex}


%--- REFERENCES / LITERATURA ---------------------------------------------------

% References are automatically generated from the supplied .bib file / Literatura se automatski generira iz zadane .bib datoteke
% Enter the name of the BibTeX file without .bib extension / Upiši ime BibTeX datoteke bez .bib nastavka
\bibliography{references}


%--- ABSTRACT / SAŽETAK --------------------------------------------------------

% Abstract in English
\begin{abstract}
  The thesis describes the design of mobile robot paths, with a focus on
  material handling vehicles, such as forklifts. It analyzes the differences
  between B\'ezier curves, B-splines and NURBS for this purpose, with the most
  important property being G3 continuity. A matrix notation of NURBS is
  described. A C++ library for the creation and usage of NURBS curves was
  developed using the Eigen library. The thesis also describes the visualization
  of parametric curves and the tools most commonly used in robotics. An
  interactive Qt desktop application was created for testing the NURBS curve
  library, and an RViz plugin was developed to be used with mobile robots in
  practice. The library and plugin are tested on a series of examples and the
  results show that the library is suitable for use in mobile robot path
  planning.
\end{abstract}

\begin{keywords}
  NURBS; mobile robots; parametric curves; Eigen; geometric continuity; Qt
\end{keywords}


% Sažetak na hrvatskom
\begin{sazetak}
  Rad opisuje dizajn putanja mobilnih robota, pogotovo vozila koja prevoze
  teret, kao viličari. Analizira razlike između B\'ezierovih krivulja, B-spline
  krivulja i NURBS krivulja u tu svrhu, s najvažnijim svojstvom da ima G3
  kontinuitet. Opisan je matrični zapis NURBS krivulja. Razvijena je C++
  biblioteka za stvaranje i korištenje NURBS krivulja korištenjem Eigen
  biblioteke. Rad također opisuje vizualizaciju parametarskih krivulja i alate
  koji se najčešće koriste u robotici. Izrađena je interaktivna Qt desktop
  aplikacija za testiranje NURBS biblioteke krivulja, a razvijen je i RViz
  dodatak za korištenje s mobilnim robotima u praksi. Biblioteka i dodatak
  testirani su na nizu primjera, te rezultati pokazuju da je biblioteka prikladna
  za korištenje u planiranju putanja mobilnih robota.
\end{sazetak}

\begin{kljucnerijeci}
  NURBS; mobilni roboti; parametarske krivulje; Eigen; geometrijski
  kontinuitet; Qt
\end{kljucnerijeci}


%--- APPENDIX / PRIVITCI -------------------------------------------------------

% All following chapters will be denoted with an appendix and a letter / Sva poglavlja koja slijede će biti označena slovom i riječi privitak
\backmatter

\chapter{NURBS.h}
\subfile{Appendix/nurbs_h.tex}

\chapter{Span.h}
\subfile{Appendix/span_h.tex}

\end{document}
